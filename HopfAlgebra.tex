\documentclass[10pt]{article}
\usepackage[utf8]{inputenc}
%\usepackage{ntheorem}
\usepackage{amsthm}
\usepackage[english]{babel}

\newtheorem{theorem}{Theorem}[section]
%\newtheorem{proof}{Proof}
\renewcommand\qedsymbol{$\blacksquare$}

\title{Test-Article}
\author{chuyendungdangki }
\date{August 2015}

\begin{document}

\maketitle
\noindent
\parindent0pt

\section{Introduction}
Let $(\mathcal{H}_1,m_1,1_{\mathcal{H}_1},\Delta_1,\epsilon_1,S_1)$ and $(\mathcal{H}_2,m_2,1_{\mathcal{H}_2},\Delta_2,\epsilon_2,S_2)$ be Hopf algebras.

Denote that $\tau_{23} (a\otimes b \otimes c \otimes d) = a\otimes c \otimes b \otimes d$. 

Then, one denotes that
\begin{itemize}
\item $m = (m_1 \otimes m_2)\circ \tau_{23}$;
\item $\Delta = \tau_{23}\circ (\Delta_1 \otimes \Delta_2)$.
\end{itemize}

Taking the tensor product $\mathcal{H}_1 \otimes \mathcal{H}_2$, one can prove that
\begin{itemize}
\item[1]  $(\mathcal{H}_1 \otimes \mathcal{H}_2,(m_1\otimes m_2)\circ \tau_{23},1_{\mathcal{H}_1} \otimes 1_{\mathcal{H}_2}$ is an algebra;
\item[2] $(\mathcal{H}_1 \otimes \mathcal{H}_2,\tau_{23} \circ (\Delta_1 \otimes \Delta_2),\epsilon_1 \otimes \epsilon_2)$ is a cogebra.
\end{itemize}

{\bf Question:} $\mathcal{H}_1 \otimes \mathcal{H}_2$ is a Hopf algebra.

\begin{theorem}
$\mathcal{H}_1 \otimes \mathcal{H}_2$ is a bialgebra.
\end{theorem}

\begin{proof}
One prove that $\Delta = \tau_{23} \circ (\Delta_1 \otimes \Delta_2)$ is an algebra morphism from $\mathcal{H}_1 \otimes \mathcal{H}_2$ to itself. It means that

\begin{equation}
\Delta \circ m = m \circ (\Delta\otimes \Delta).
\end{equation}
\end{proof}

\end{document}